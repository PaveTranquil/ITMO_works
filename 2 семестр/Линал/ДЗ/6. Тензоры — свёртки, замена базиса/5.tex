\documentclass{article}
\usepackage{mathtext}
\usepackage[russian]{babel}
\usepackage[a3paper, paperwidth=28cm, paperheight=14cm, top=1cm,bottom=1cm,left=1cm,right=1cm,marginparwidth=1.75cm]{geometry}
\usepackage{amsmath}
\usepackage{amssymb}
\usepackage{multicol}
\usepackage{fancyhdr}
\usepackage{nicefrac}
\usepackage{graphicx}
\usepackage{cancel}
\usepackage{wrapfig}
\usepackage{tikz}
\pagenumbering{gobble}


\newlength{\tempheight}
\newcommand{\Let}[0]{%
\mathbin{\text{\settoheight{\tempheight}{\mathstrut}\raisebox{0.5\pgflinewidth}{%
\tikz[baseline,line cap=round,line join=round] \draw (0,0) --++ (0.4em,0) --++ (0,1.5ex) --++ (-0.4em,0);%
}}}\;}
\newcommand{\e}{\text{e}}
\newcommand{\la}{\lambda}
\newcommand{\shiftleft}[3]{\makebox[#1][r]{\makebox[#2][l]{#3}}}
\newcommand{\shiftright}[3]{\makebox[#2][r]{\makebox[#1][l]{#3}}}
\newcommand*\circled[1]{\tikz[baseline=(char.base)]{
            \node[shape=circle,draw,inner sep=2pt] (char) {#1};}}
\newcommand*\squared[1]{\tikz[baseline=(char.base)]{
            \node[shape=rectangle,draw,inner sep=4pt] (char) {#1};}}
\newcommand{\at}{\biggr\rvert}

\begin{document}
\begin{center}
Свёртка по $n$ будет выглядеть как $c^{nk}_n = a^{1k}_1 + a^{2k}_2 + a^{3k}_3$. \\
Отобразим свёртку в тензорном виде:
$$c^k = \begin{Vmatrix}
a^{11}_1 \\ \, \\ a^{12}_1 \\ \, \\ a^{13}_1
\end{Vmatrix} + \begin{Vmatrix}
a^{21}_2 \\ \, \\ a^{22}_2 \\ \, \\ a^{23}_2
\end{Vmatrix} + \begin{Vmatrix}
a^{31}_3 \\ \, \\ a^{32}_3 \\ \, \\ a^{33}_3
\end{Vmatrix}$$
Сопоставим каждому из компонентов соответствующий компонент исходного тензора и получим:
$$c^k =\begin{Vmatrix}
-3 \\ 1 \\ -1
\end{Vmatrix} + \begin{Vmatrix}
4 \\ 0 \\ -2
\end{Vmatrix} + \begin{Vmatrix}
-3 \\ 0 \\ -2
\end{Vmatrix} = \begin{Vmatrix}
1 \\ 1 \\ -3
\end{Vmatrix} + \begin{Vmatrix}
-3 \\ 0 \\ -2
\end{Vmatrix} = \begin{Vmatrix}
-2 \\ 1 \\ -5
\end{Vmatrix}$$ 
\end{center}
\end{document}