\documentclass{article}
\usepackage{mathtext}
\usepackage[russian]{babel}
\usepackage[a3paper, paperwidth=28cm, paperheight=14cm, top=1cm,bottom=1cm,left=1cm,right=1cm,marginparwidth=1.75cm]{geometry}
\usepackage{amsmath}
\usepackage{amssymb}
\usepackage{multicol}
\usepackage{fancyhdr}
\usepackage{nicefrac}
\usepackage{graphicx}
\usepackage{cancel}
\usepackage{wrapfig}
\usepackage{tikz}
\pagenumbering{gobble}


\newlength{\tempheight}
\newcommand{\Let}[0]{%
\mathbin{\text{\settoheight{\tempheight}{\mathstrut}\raisebox{0.5\pgflinewidth}{%
\tikz[baseline,line cap=round,line join=round] \draw (0,0) --++ (0.4em,0) --++ (0,1.5ex) --++ (-0.4em,0);%
}}}\;}
\newcommand{\e}{\text{e}}
\newcommand{\la}{\lambda}
\newcommand{\shiftleft}[3]{\makebox[#1][r]{\makebox[#2][l]{#3}}}
\newcommand{\shiftright}[3]{\makebox[#2][r]{\makebox[#1][l]{#3}}}
\newcommand*\circled[1]{\tikz[baseline=(char.base)]{
            \node[shape=circle,draw,inner sep=2pt] (char) {#1};}}
\newcommand*\squared[1]{\tikz[baseline=(char.base)]{
            \node[shape=rectangle,draw,inner sep=4pt] (char) {#1};}}
\newcommand{\at}{\biggr\rvert}

\begin{document}
\begin{center}
Определим свёртку по $n$ как $b^t_t = a^{1t}_{t1} + a^{2t}_{t2} + a^{3t}_{t3}$. \\
Тогда свёртка по $t$ будет выглядеть как $c = b^1_1 + b^2_2 + b^3_3 = \left (a^{11}_{11} + a^{21}_{12} + a^{31}_{13}\right ) + \left (a^{12}_{21} + a^{22}_{22} + a^{32}_{23}\right ) + \left (a^{13}_{31} + a^{23}_{32} + a^{33}_{33}\right )$. \\
Сопоставим каждому из компонентов соответствующее значение исходного тензора и получим:
$$c = (1 + 0 + 0) + (-4 - 2 + 1) + (2 + 2 - 1) = 1 - 5 + 3 = \Vert-1\Vert$$ 
\end{center}
\end{document}