\documentclass{article}
\usepackage{mathtext}
\usepackage[russian]{babel}
\usepackage[a3paper, paperwidth=28cm, paperheight=14cm, top=1cm,bottom=1cm,left=1cm,right=1cm,marginparwidth=1.75cm]{geometry}
\usepackage{amsmath}
\usepackage{amssymb}
\usepackage{multicol}
\usepackage{fancyhdr}
\usepackage{nicefrac}
\usepackage{graphicx}
\usepackage{cancel}
\usepackage{wrapfig}
\usepackage{tikz}
\pagenumbering{gobble}


\newlength{\tempheight}
\newcommand{\Let}[0]{%
\mathbin{\text{\settoheight{\tempheight}{\mathstrut}\raisebox{0.5\pgflinewidth}{%
\tikz[baseline,line cap=round,line join=round] \draw (0,0) --++ (0.4em,0) --++ (0,1.5ex) --++ (-0.4em,0);%
}}}\;}
\newcommand{\e}{\text{e}}
\newcommand{\la}{\lambda}
\newcommand{\shiftleft}[3]{\makebox[#1][r]{\makebox[#2][l]{#3}}}
\newcommand{\shiftright}[3]{\makebox[#2][r]{\makebox[#1][l]{#3}}}
\newcommand*\circled[1]{\tikz[baseline=(char.base)]{
            \node[shape=circle,draw,inner sep=2pt] (char) {#1};}}
\newcommand*\squared[1]{\tikz[baseline=(char.base)]{
            \node[shape=rectangle,draw,inner sep=4pt] (char) {#1};}}
\newcommand{\at}{\biggr\rvert}

\begin{document}
\begin{center}
Результатом симметризации по трём индексам будет тензор $b^{kpi} = \dfrac{1}{3!}\left (a^{kpi} + a^{kip} + a^{pki} + a^{pik} + a^{ipk} + a^{ikp}\right )$. \\
Теперь переберём каждый индекс, чтобы найти компоненты результирующего тензора:
\begin{gather*}
b^{111} = a^{111} = -1 \\
b^{112} = b^{121} = b^{211} = \frac{1}{3}\left(a^{112} + a^{121} + a^{211}\right) = \frac{1}{3}(6 - 2 + 3) = \frac{7}{3} \\
b^{122} = b^{212} = b^{221} = \frac{1}{3}\left(a^{122} + a^{212} + a^{221}\right) = \frac{1}{3}(-1 - 1 - 5) = -\frac{7}{3} \\
b^{222} = a^{222} = 0
\end{gather*}
Итак, тензор $b^{kpi}$ будет определяться матрицей $B$:
$$B = \begin{array}{||cc|cc||}
-1 & 2.(3) & 2.(3) & -2.(3) \\
2.(3) & -2.(3) & -2.(3) & 0
\end{array}$$
\end{center}
\end{document}