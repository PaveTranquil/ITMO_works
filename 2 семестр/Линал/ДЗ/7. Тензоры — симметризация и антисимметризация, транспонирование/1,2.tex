\documentclass{article}
\usepackage{mathtext}
\usepackage[russian]{babel}
\usepackage[a3paper, paperwidth=28cm, paperheight=14cm, top=1cm,bottom=1cm,left=1cm,right=1cm,marginparwidth=1.75cm]{geometry}
\usepackage{amsmath}
\usepackage{amssymb}
\usepackage{multicol}
\usepackage{fancyhdr}
\usepackage{nicefrac}
\usepackage{graphicx}
\usepackage{cancel}
\usepackage{wrapfig}
\usepackage{tikz}
\pagenumbering{gobble}


\newlength{\tempheight}
\newcommand{\Let}[0]{%
\mathbin{\text{\settoheight{\tempheight}{\mathstrut}\raisebox{0.5\pgflinewidth}{%
\tikz[baseline,line cap=round,line join=round] \draw (0,0) --++ (0.4em,0) --++ (0,1.5ex) --++ (-0.4em,0);%
}}}\;}
\newcommand{\e}{\text{e}}
\newcommand{\la}{\lambda}
\newcommand{\shiftleft}[3]{\makebox[#1][r]{\makebox[#2][l]{#3}}}
\newcommand{\shiftright}[3]{\makebox[#2][r]{\makebox[#1][l]{#3}}}
\newcommand*\circled[1]{\tikz[baseline=(char.base)]{
            \node[shape=circle,draw,inner sep=2pt] (char) {#1};}}
\newcommand*\squared[1]{\tikz[baseline=(char.base)]{
            \node[shape=rectangle,draw,inner sep=4pt] (char) {#1};}}
\newcommand{\at}{\biggr\rvert}

\begin{document}
\begin{center}
Транспонирование тензора вида $b^{ijk} = a^{jik}$, т.е. транспонирование с закреплением слоёв означает, \\
что внутри каждого слоя происходит стандартное транспонирование по строкам и столбцам, \\
потому поиск компонентов тензора $b$ становится тривиальной задачей. Например: \\ \, \\

Пусть $A$: $a^{jl}_k$ и $B$: $b^{jl}_k = a^{lj}_k$, тогда:
$$A = \begin{array}{||cc|cc||}
0 & 4 & 1 & 2 \\
1 & 3 & 0 & 1
\end{array} \Rightarrow B = \begin{array}{||cc|cc||}
0 & 1 & 1 & 0 \\
4 & 3 & 2 & 1
\end{array}$$
\end{center}
\end{document}