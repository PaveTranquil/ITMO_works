\documentclass{article}
\usepackage{cmap}
\usepackage{mathtext}
\usepackage[russian]{babel}
\usepackage[a3paper, paperwidth=28cm, paperheight=14cm, top=1cm,bottom=1cm,left=1cm,right=1cm,marginparwidth=1.75cm]{geometry}
\usepackage{amsmath}
\usepackage{amssymb}
\usepackage{multicol}
\usepackage{fancyhdr}
\usepackage{nicefrac}
\usepackage{graphicx}
\usepackage{cancel}
\usepackage{wrapfig}
\usepackage{tikz}
\pagenumbering{gobble}


\newlength{\tempheight}
\newcommand{\Let}[0]{%
\mathbin{\text{\settoheight{\tempheight}{\mathstrut}\raisebox{0.5\pgflinewidth}{%
\tikz[baseline,line cap=round,line join=round] \draw (0,0) --++ (0.4em,0) --++ (0,1.5ex) --++ (-0.4em,0);%
}}}\;}
\newcommand{\e}{\text{e}}
\newcommand{\la}{\lambda}
\newcommand{\shiftleft}[3]{\makebox[#1][r]{\makebox[#2][l]{#3}}}
\newcommand{\shiftright}[3]{\makebox[#2][r]{\makebox[#1][l]{#3}}}
\newcommand*\circled[1]{\tikz[baseline=(char.base)]{
            \node[shape=circle,draw,inner sep=2pt] (char) {#1};}}
\newcommand*\squared[1]{\tikz[baseline=(char.base)]{
            \node[shape=rectangle,draw,inner sep=4pt] (char) {#1};}}
\newcommand{\at}{\biggr\rvert}

\begin{document}
\begin{center}
    Для того, чтобы найти угол между векторами, для начала нужно найти косинус этого угла, который вычисляется по следующей формуле:
    $$\cos\varphi = \frac{x^{T}Gy}{\sqrt{x^{T}Gx}\sqrt{y^{T}Gy}}$$
    Разделим на компоненты и посчитаем каждый отдельно:
    $$x^{T}Gy = \begin{pmatrix}
            0 & -2 & 2 & 6
        \end{pmatrix}\times\begin{pmatrix}
            6 & 3 & 1 & 5 \\
            3 & 2 & 1 & 3 \\
            1 & 1 & 2 & 1 \\
            5 & 3 & 1 & 5
        \end{pmatrix}\times\begin{pmatrix}
            -2 \\ 4 \\ 0 \\ -6
        \end{pmatrix} = \begin{pmatrix}
            26 & 16 & 8 & 26
        \end{pmatrix} \times \begin{pmatrix}
            -2 \\ 4 \\ 0 \\ -6
        \end{pmatrix} = -144$$
    $$x^{T}Gx = \begin{pmatrix}
            0 & -2 & 2 & 6
        \end{pmatrix}\times\begin{pmatrix}
            6 & 3 & 1 & 5 \\
            3 & 2 & 1 & 3 \\
            1 & 1 & 2 & 1 \\
            5 & 3 & 1 & 5
        \end{pmatrix}\times\begin{pmatrix}
            0 \\ -2 \\ 2 \\ 6
        \end{pmatrix} = \begin{pmatrix}
            26 & 16 & 8 & 26
        \end{pmatrix} \times \begin{pmatrix}
            0 \\ -2 \\ 2 \\ 6
        \end{pmatrix} = 140$$
    $$y^{T}Gy = \begin{pmatrix}
            -2 & 4 & 0 & -6
        \end{pmatrix}\times\begin{pmatrix}
            6 & 3 & 1 & 5 \\
            3 & 2 & 1 & 3 \\
            1 & 1 & 2 & 1 \\
            5 & 3 & 1 & 5
        \end{pmatrix}\times\begin{pmatrix}
            -2 \\ 4 \\ 0 \\ -6
        \end{pmatrix} = \begin{pmatrix}
            -30 & -16 & -4 & -28
        \end{pmatrix} \times \begin{pmatrix}
            -2 \\ 4 \\ 0 \\ -6
        \end{pmatrix} = 164$$
    Посчитаем косинус, подставив найденные выше значения:
    $$\cos\varphi = \frac{-144}{\sqrt{140}\sqrt{164}} = -\frac{36\sqrt{1435}}{1435} \Rightarrow \varphi = \pi - \arccos\frac{36\sqrt{1435}}{1435} \approx 2.83$$
\end{center}
\end{document}