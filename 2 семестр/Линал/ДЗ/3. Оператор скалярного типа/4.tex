\documentclass{article}
\usepackage{mathtext}
\usepackage[russian]{babel}
\usepackage[a3paper, paperwidth=28cm, paperheight=28cm, top=1cm,bottom=1cm,left=1cm,right=1cm,marginparwidth=1.75cm]{geometry}
\usepackage{amsmath}
\usepackage{amssymb}
\usepackage{graphicx}
\usepackage{cancel}
\usepackage{wrapfig}
\pagenumbering{gobble}
\newcommand{\e}{\text{e}}

\begin{document}
\begin{center}
Найдём собственные значения оператора:
\end{center}
$$\left|\begin{matrix}
4-\lambda & 0 & 0 & 9 \\
0 & 4-\lambda & 0 & -9 \\
0 & 0 & 4-\lambda & 9 \\
0 & 0 & 0 & -5-\lambda
\end{matrix}\right| = 0$$
$$\lambda^4-7\lambda^3-12\lambda^2+176\lambda-320 = 0$$
$$(\lambda+5)(\lambda-4)^3=0$$
$$\Downarrow$$
$$\sigma_A = \{-5^{(1)},4^{(3)}\}$$ \\
\begin{center}
Найдём собственные вектора:
\end{center}
$$\lambda = -5$$
$$\begin{pmatrix}
9 & 0 & 0 & 9 \\
0 & 9 & 0 & -9 \\
0 & 0 & 9 & 9 \\
0 & 0 & 0 & 0
\end{pmatrix} \Rightarrow \begin{cases}
x_1 = -x_4 \\
x_2 = x_4\\
x_3 = -x_4 \\
x_4 \in \mathbb{R}
\end{cases} \Rightarrow v_1 = \begin{pmatrix}
-1 \\ 1 \\ -1 \\ 1
\end{pmatrix}$$\\
$$\lambda = 4$$
$$\begin{pmatrix}
0 & 0 & 0 & 9 \\
0 & 0 & 0 & -9 \\
0 & 0 & 0 & 9 \\
0 & 0 & 0 & -9
\end{pmatrix} \sim \begin{pmatrix}
0 & 0 & 0 & 1 \\
0 & 0 & 0 & 0 \\
0 & 0 & 0 & 0 \\
0 & 0 & 0 & 0
\end{pmatrix} \Rightarrow \begin{cases}
x_1 \in \mathbb{R} \\
x_2 \in \mathbb{R} \\
x_3 \in \mathbb{R} \\
x_4 = 0 \\
\end{cases} \Rightarrow v_2 = \begin{pmatrix}
1 \\ 0 \\ 0 \\ 0
\end{pmatrix} \quad v_3 = \begin{pmatrix}
0 \\ 1 \\ 0 \\ 0
\end{pmatrix} \quad v_4 = \begin{pmatrix}
0 \\ 0 \\ 1 \\ 0
\end{pmatrix}$$ \\
\begin{center}
Алгебраическая и геометрическая кратности значений спектра совпадают, а значит мы уже можем построить Жорданову форму, или диагональную матрицу оператора в базисе $\{v_1, v_2, v_3, v_4\}$, или $\hat{A}$:
\end{center}
$$\hat{A} = \begin{pmatrix}
-5 & 0 & 0 & 0 \\
0 & 4 & 0 & 0 \\
0 & 0 & 4 & 0 \\
0 & 0 & 0 & 4
\end{pmatrix}$$
\end{document}