\documentclass{article}
\usepackage{cmap}
\usepackage{mathtext}
\usepackage[russian]{babel}
\usepackage[a3paper, paperwidth=28cm, paperheight=14cm, top=1cm,bottom=1cm,left=1cm,right=1cm,marginparwidth=1.75cm]{geometry}
\usepackage{amsmath}
\usepackage{amssymb}
\usepackage{multicol}
\usepackage{fancyhdr}
\usepackage{nicefrac}
\usepackage{graphicx}
\usepackage{cancel}
\usepackage{wrapfig}
\usepackage{tikz}
\pagenumbering{gobble}


\newlength{\tempheight}
\newcommand{\Let}[0]{%
\mathbin{\text{\settoheight{\tempheight}{\mathstrut}\raisebox{0.5\pgflinewidth}{%
\tikz[baseline,line cap=round,line join=round] \draw (0,0) --++ (0.4em,0) --++ (0,1.5ex) --++ (-0.4em,0);%
}}}\;}
\newcommand{\e}{\text{e}}
\newcommand{\la}{\lambda}
\newcommand{\shiftleft}[3]{\makebox[#1][r]{\makebox[#2][l]{#3}}}
\newcommand{\shiftright}[3]{\makebox[#2][r]{\makebox[#1][l]{#3}}}
\newcommand*\circled[1]{\tikz[baseline=(char.base)]{
            \node[shape=circle,draw,inner sep=2pt] (char) {#1};}}
\newcommand*\squared[1]{\tikz[baseline=(char.base)]{
            \node[shape=rectangle,draw,inner sep=4pt] (char) {#1};}}
\newcommand{\at}{\biggr\rvert}

\begin{document}
\begin{center}
    Решим систему уравнений и таким образом найдём базис пространства:
    $$
        \begin{cases}
            -x_1+2x_2-2x_3-2x_4 = 0 \\
            -x_1+4x_2-6x_3-6x_4 = 0 \\
            -x_2+2x_3+2x_4 = 0 \\
            -x_1+4x_2-7x_3-7x_4 = 0
        \end{cases} \Rightarrow e = \left\{\begin{pmatrix}
            0 \\ 0 \\ -1 \\ 1
        \end{pmatrix}\right\}
    $$
    Найдём ортогональную проекцию:
    $$
        \frac{e_1^{T}Gy}{e_1^TGe_1}e_1  = \frac{-205}{170}e_1 = \begin{pmatrix}
            0 \\ 0 \\ \nicefrac{41}{34} \\ \nicefrac{-41}{34}
        \end{pmatrix}
    $$
\end{center}
\end{document}