\documentclass{article}
\usepackage{mathtext}
\usepackage[russian]{babel}
\usepackage[a3paper, paperwidth=28cm, paperheight=14cm, top=1cm,bottom=1cm,left=1cm,right=1cm,marginparwidth=1.75cm]{geometry}
\usepackage{amsmath}
\usepackage{amssymb}
\usepackage{graphicx}
\usepackage{cancel}
\usepackage{wrapfig}
\pagenumbering{gobble}

\newlength{\tempheight}
\newcommand{\Let}[0]{%
\mathbin{\text{\settoheight{\tempheight}{\mathstrut}\raisebox{0.5\pgflinewidth}{%
\tikz[baseline,line cap=round,line join=round] \draw (0,0) --++ (0.4em,0) --++ (0,1.5ex) --++ (-0.4em,0);%
}}}\;}

\begin{document}
\begin{center}
Получим решение воздействием оператора на вектора линейной оболочки и нахождением базиса из получившихся векторов.
$$\begin{pmatrix}
-6 & 15 & 6 & 18 & -9 \\ 12 & -30 & -12 & -36 & 18 \\ 3 & -6 & -3 & -9 & 3
\end{pmatrix}\times \begin{pmatrix}
-3 \\ 7 \\ -8 \\ -15 \\ -13
\end{pmatrix} = \begin{pmatrix}
-78 & 156 & 69
\end{pmatrix}^{T}$$
$$\begin{pmatrix}
-6 & 15 & 6 & 18 & -9 \\ 12 & -30 & -12 & -36 & 18 \\ 3 & -6 & -3 & -9 & 3
\end{pmatrix}\times \begin{pmatrix}
-1 \\ 2 \\ -2 \\ -4 \\ -3
\end{pmatrix} = \begin{pmatrix}
-21 & 42 & 18
\end{pmatrix}^{T}$$
$$\begin{pmatrix}
-6 & 15 & 6 & 18 & -9 \\ 12 & -30 & -12 & -36 & 18 \\ 3 & -6 & -3 & -9 & 3
\end{pmatrix}\times \begin{pmatrix}
2 \\ -5 \\ 5 \\ 10 \\ 9
\end{pmatrix} = \begin{pmatrix}
42 & -84 & -42
\end{pmatrix}^{T}$$ \\
$$\begin{pmatrix}
-78 & 156 & 69 \\ -21 & 42 & 18 \\ 42 & -84 & -42
\end{pmatrix} \sim \begin{pmatrix}
-26 & 52 & 23 \\ -7 & 14 & 6 \\ 1 & -2 & -1
\end{pmatrix} \sim \begin{pmatrix}
26 & 52 & -23 \\ 0 & 0 & 1 \\ 0 & 0 & 0
\end{pmatrix}$$
\end{center}
\end{document}