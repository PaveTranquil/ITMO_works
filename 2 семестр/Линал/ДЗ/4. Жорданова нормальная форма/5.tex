\documentclass{article}
\usepackage{mathtext}
\usepackage[russian]{babel}
\usepackage[a3paper, paperwidth=30cm, paperheight=17cm, top=1cm,bottom=1cm,left=1cm,right=1cm,marginparwidth=1.75cm]{geometry}
\usepackage{amsmath}
\usepackage{amssymb}
\usepackage{multicol}
\usepackage{fancyhdr}
\usepackage{nicefrac}
\usepackage{graphicx}
\usepackage{cancel}
\usepackage{wrapfig}
\usepackage{tikz}
\pagenumbering{gobble}


\newlength{\tempheight}
\newcommand{\Let}[0]{%
\mathbin{\text{\settoheight{\tempheight}{\mathstrut}\raisebox{0.5\pgflinewidth}{%
\tikz[baseline,line cap=round,line join=round] \draw (0,0) --++ (0.4em,0) --++ (0,1.5ex) --++ (-0.4em,0);%
}}}\;}
\newcommand{\e}{\text{e}}
\newcommand{\la}{\lambda}
\newcommand{\shiftleft}[3]{\makebox[#1][r]{\makebox[#2][l]{#3}}}
\newcommand{\shiftright}[3]{\makebox[#2][r]{\makebox[#1][l]{#3}}}
\newcommand*\circled[1]{\tikz[baseline=(char.base)]{
            \node[shape=circle,draw,inner sep=2pt] (char) {#1};}}
\newcommand*\squared[1]{\tikz[baseline=(char.base)]{
            \node[shape=rectangle,draw,inner sep=4pt] (char) {#1};}}
\newcommand{\at}{\biggr\rvert}

\begin{document}
\begin{center}
Вычислим характеристический полином и, как следствие, найдём спектр оператора:
$$\chi_\varphi(\la) = \begin{vmatrix}
4-\la & 4 & 2 & 1 \\
-49 & -28-\la & -16 & -8 \\
45 & 22 & 15-\la & 5 \\
57 & 28 & 14 & 12-\la \\
\end{vmatrix} = \la^4-3\la^3-25\la^2+39\la+180 = (\la - 5)(\la - 4)(\la + 3)^2\Rightarrow \sigma_\varphi = \left\{4^{(1)}, 5^{(1)}, -3^{(2)}\right\}$$
Найдём собственные вектора оператора к соответствующим собственным значениям:
$$\la = 4$$
$$\begin{pmatrix}
0 & 4 & 2 & 1 \\
-49 & -32 & -16 & -8 \\
45 & 22 & 11 & 5 \\
57 & 28 & 14 & 8 \\
\end{pmatrix} \sim \begin{pmatrix}
0 & 4 & 2 & 1 \\
4 & 10 & 5 & 3 \\
45 & 22 & 11 & 5 \\
12 & 6 & 3 & 3 \\
\end{pmatrix} \sim \begin{pmatrix}
0 & 4 & 2 & 1 \\
0 & 8 & 4 & 2 \\
45 & 22 & 11 & 5 \\
4 & 2 & 1 & 1 \\
\end{pmatrix} \sim \begin{pmatrix}
0 & 4 & 2 & 1 \\
0 & 0 & 0 & 0 \\
1 & 0 & 0 & -6 \\
4 & 2 & 1 & 1 \\
\end{pmatrix} \sim \begin{pmatrix}
0 & 0 & 0 & 1 \\
0 & 0 & 0 & 0 \\
1 & 0 & 0 & -6 \\
0 & 2 & 1 & 25 \\
\end{pmatrix} \Rightarrow \begin{cases}
\xi_1 = \xi_4 = 0 \\
\xi_3 = -2\xi_2 \\
\xi_2 \in \mathbb{R}
\end{cases} \shiftleft{3pt}{14pt}{$\Rightarrow$} v_1 = \begin{pmatrix}
0 \\ 1 \\ -2 \\ 0
\end{pmatrix}$$
$$\la = 5$$
$$\begin{pmatrix}
-1 & 4 & 2 & 1 \\
-49 & -33 & -16 & -8 \\
45 & 22 & 10 & 5 \\
57 & 28 & 14 & 7 \\
\end{pmatrix} \sim \begin{pmatrix}
-1 & 4 & 2 & 1 \\
4 & 11 & 6 & 3 \\
45 & 22 & 10 & 5 \\
12 & 6 & 4 & 2 \\
\end{pmatrix} \sim \begin{pmatrix}
-1 & 4 & 2 & 1 \\
-2 & 8 & 4 & 2 \\
3 & 2 & 6 & 3 \\
6 & 3 & 2 & 1 \\
\end{pmatrix} \sim \begin{pmatrix}
-1 & 4 & 2 & 1 \\
0 & 0 & 0 & 0 \\
0 & 14 & 12 & 6 \\
0 & 1 & 10 & 5 \\
\end{pmatrix} \sim \begin{pmatrix}
-1 & 4 & 2 & 1 \\
0 & 0 & 0 & 0 \\
0 & 1 & 0 & 0 \\
0 & 1 & 10 & 5 \\
\end{pmatrix} \Rightarrow \begin{cases}
\xi_1 = \xi_2 = 0 \\
\xi_4 = -2\xi_3 \\
\xi_3 \in \mathbb{R}
\end{cases}  \shiftleft{0pt}{14pt}{$\Rightarrow$} v_2=\begin{pmatrix}
0 \\ 0 \\ 1 \\ -2
\end{pmatrix}$$
$$\la = -3$$
$$\begin{pmatrix}
7 & 4 & 2 & 1 \\
-49 & -25 & -16 & -8 \\
45 & 22 & 18 & 5 \\
57 & 28 & 14 & 15 \\
\end{pmatrix} \sim \begin{pmatrix}
7 & 4 & 2 & 1 \\
0 & 3 & -2 & -1 \\
-4 & -3 & 2 & -3 \\
8 & 3 & -2 & 7 \\
\end{pmatrix} \sim \begin{pmatrix}
7 & 7 & 0 & 0 \\
0 & -3 & 2 & 1 \\
4 & 0 & 0 & 4 \\
4 & 0 & 0 & 4 \\
\end{pmatrix} \sim \begin{pmatrix}
1 & 1 & 0 & 0 \\
0 & -3 & 2 & 1 \\
1 & 0 & 0 & 1 \\
0 & 0 & 0 & 0 \\
\end{pmatrix} \sim \begin{pmatrix}
0 & 1 & 0 & -1 \\
0 & 0 & 1 & -1 \\
1 & 0 & 0 & 1 \\
0 & 0 & 0 & 0 \\
\end{pmatrix} \Rightarrow \begin{cases}
\xi_1 = -\xi_4 \\
\xi_2 = \xi_4 \\
\xi_3 = \xi_4
\xi_4 \in \mathbb{R}
\end{cases}  \shiftleft{-10pt}{14pt}{$\Rightarrow$} v_3=\begin{pmatrix}
-1 \\ 1 \\ 1 \\ 1
\end{pmatrix}$$
Для этого собственного значения алгебраическая и геометрическая кратности не совпадают. Это означает, что мы не сможем построить базис оператора, \\в котором будет существовать жорданова нормальная форма для матрицы оператора. Найдём присоединённый вектор для $v_3$:
$$\left(\begin{array}{cccc|c}
7 & 4 & 2 & 1 & -1\\
-49 & -25 & -16 & -8 & 1\\
45 & 22 & 18 & 5 & 1\\
57 & 28 & 14 & 15 & 1\\
\end{array}\right) \sim \left(\begin{array}{cccc|c}
7 & 4 & 2 & 1 & -1\\
0 & 3 & -2 & -1 & -6\\
4 & 3 & -2 & 3 & -2\\
8 & 3 & -2 & 7 & 2\\
\end{array}\right) \sim \left(\begin{array}{cccc|c}
7 & 7 & 0 & 0 & -7\\
0 & -3 & 2 & 1 & 6\\
4 & 0 & 0 & 4 & 4\\
4 & 0 & 0 & 4 & 4\\
\end{array}\right) \sim \left(\begin{array}{cccc|c}
1 & 1 & 0 & 0 & -1\\
0 & -3 & 2 & 1 & 6\\
1 & 0 & 0 & 1 & 1\\
0 & 0 & 0 & 0 & 0\\
\end{array}\right) \sim \left(\begin{array}{cccc|c}
0 & 1 & 0 & -1 & -2\\
0 & 0 & 1 & -1 & 0\\
1 & 0 & 0 & 1 & 1\\
0 & 0 & 0 & 0 & 0\\
\end{array}\right) \Rightarrow \begin{cases}
\xi_1 = 1 - \xi_4 \\
\xi_2 = \xi_4 - 2 \\
\xi_3 = \xi_4 \\
\xi_4 \in \mathbb{R}
\end{cases} \shiftleft{5pt}{14pt}{$\Rightarrow$} u_1 = \begin{pmatrix}
0 \\ -1 \\ 1 \\ 1
\end{pmatrix}$$
\end{center}
\end{document}