\documentclass{article}
\usepackage[russian]{babel}
\usepackage[a3paper, paperwidth=21cm, paperheight=14cm, top=1cm,bottom=1cm,left=1cm,right=1cm,marginparwidth=1.75cm]{geometry}
\usepackage{amsmath}
\usepackage{amssymb}
\usepackage{graphicx}
\usepackage{cancel}
\usepackage{wrapfig}
\pagenumbering{gobble}

\begin{document}
\begin{center}
Подействуем оператором на стандартный базис:
\end{center}
\noindent$$\phi(\left(\begin{array}{c}\xi^1 \\ \xi^2\end{array}\right)) = \left(\begin{array}{c}\xi^1 - \xi^2 \\ \xi^2\end{array}\right) \Rightarrow
\left[\phi(\left(\begin{array}{c}1 \\ 0\end{array}\right)),\; \phi(\left(\begin{array}{c}0 \\ 1\end{array}\right))\right] =
\left[\begin{array}{rr}
1 & -1 \\ 0 & 1
\end{array}\right]$$
\begin{center}
Матрицу линейного оператора так же нетрудно построить, если оператор действует на элемент $x = (\xi^1, \xi^2)^T$, в котором нет лишних коэффициентов. Для этого выписываем коэффициенты перед значениями $\xi^1$ и $\xi^2$ в операторе. 
\end{center}
\end{document}