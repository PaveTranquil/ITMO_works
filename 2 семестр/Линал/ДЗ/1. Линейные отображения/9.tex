\documentclass{article}
\usepackage[russian]{babel}
\usepackage[a3paper, paperwidth=21cm, paperheight=14cm, top=1cm,bottom=1cm,left=1cm,right=1cm,marginparwidth=1.75cm]{geometry}
\usepackage{amsmath}
\usepackage{amssymb}
\usepackage{graphicx}
\usepackage{cancel}
\usepackage{wrapfig}
\pagenumbering{gobble}

\begin{document}
\begin{center}
Базис ядра оператора с заданной матрицей определяется как ФСР однородной системы в матрице оператора:
\end{center}
\noindent$$\left(\begin{array}{llll}1 & 1 & 0 & -2 \\ -1 & 0 & 0 & 1 \\ 1 & 1 & 0 & -2 \\ 2^{-1} & 1 & 0 & -3^{+1}\end{array}\right) \sim \left(\begin{array}{rrrr}1 & 1 & 0 & -2 \\ -1 & 0 & 0 & 1 \\ 1 & 1 & 0 & -2 \\ 1 & 1 & 0 & -2\end{array}\right) \sim \left(\begin{array}{lrrl}-1 & 0 & 0 & 1 \\ 1^{-1} & 1 & 0 & -2^{+1} \\0 & 0 & 0 & 0 \\ 0 & 0 & 0 & 0\end{array}\right) \Leftrightarrow \begin{cases}
\xi^1= \xi^4 \\
\xi^2 =  \xi^4 \\
\xi^3, \xi^4 \in \mathbb{R}
\end{cases} \Rightarrow v_1 = \begin{pmatrix}1 \\ 1 \\ 0 \\ 1\end{pmatrix};\; v_2 = \begin{pmatrix}0 \\ 0 \\ 1 \\ 0\end{pmatrix}$$
\end{document}